\documentclass[]{article}
\usepackage{lmodern}
\usepackage{amssymb,amsmath}
\usepackage{ifxetex,ifluatex}
\usepackage{fixltx2e} % provides \textsubscript
\ifnum 0\ifxetex 1\fi\ifluatex 1\fi=0 % if pdftex
  \usepackage[T1]{fontenc}
  \usepackage[utf8]{inputenc}
\else % if luatex or xelatex
  \ifxetex
    \usepackage{mathspec}
  \else
    \usepackage{fontspec}
  \fi
  \defaultfontfeatures{Ligatures=TeX,Scale=MatchLowercase}
    \setmainfont[]{Minion Pro}
    \setsansfont[]{ITC Slimbach Std Book}
    \setmonofont[Mapping=tex-ansi]{Source Code Pro}
\fi
% use upquote if available, for straight quotes in verbatim environments
\IfFileExists{upquote.sty}{\usepackage{upquote}}{}
% use microtype if available
\IfFileExists{microtype.sty}{%
\usepackage{microtype}
\UseMicrotypeSet[protrusion]{basicmath} % disable protrusion for tt fonts
}{}
\usepackage[margin=1in]{geometry}
\usepackage{hyperref}
\PassOptionsToPackage{usenames,dvipsnames}{color} % color is loaded by hyperref
\hypersetup{unicode=true,
            pdftitle={Some Theory Underlying Simple Linear Regression},
            colorlinks=true,
            linkcolor=Maroon,
            citecolor=Blue,
            urlcolor=umn2,
            breaklinks=true}
\urlstyle{same}  % don't use monospace font for urls
\usepackage{graphicx,grffile}
\makeatletter
\def\maxwidth{\ifdim\Gin@nat@width>\linewidth\linewidth\else\Gin@nat@width\fi}
\def\maxheight{\ifdim\Gin@nat@height>\textheight\textheight\else\Gin@nat@height\fi}
\makeatother
% Scale images if necessary, so that they will not overflow the page
% margins by default, and it is still possible to overwrite the defaults
% using explicit options in \includegraphics[width, height, ...]{}
\setkeys{Gin}{width=\maxwidth,height=\maxheight,keepaspectratio}
\IfFileExists{parskip.sty}{%
\usepackage{parskip}
}{% else
\setlength{\parindent}{0pt}
\setlength{\parskip}{6pt plus 2pt minus 1pt}
}
\setlength{\emergencystretch}{3em}  % prevent overfull lines
\providecommand{\tightlist}{%
  \setlength{\itemsep}{0pt}\setlength{\parskip}{0pt}}
\setcounter{secnumdepth}{0}
% Redefines (sub)paragraphs to behave more like sections
\ifx\paragraph\undefined\else
\let\oldparagraph\paragraph
\renewcommand{\paragraph}[1]{\oldparagraph{#1}\mbox{}}
\fi
\ifx\subparagraph\undefined\else
\let\oldsubparagraph\subparagraph
\renewcommand{\subparagraph}[1]{\oldsubparagraph{#1}\mbox{}}
\fi

%%% Use protect on footnotes to avoid problems with footnotes in titles
\let\rmarkdownfootnote\footnote%
\def\footnote{\protect\rmarkdownfootnote}

%%% Change title format to be more compact
\usepackage{titling}

% Create subtitle command for use in maketitle
\newcommand{\subtitle}[1]{
  \posttitle{
    \begin{center}\large#1\end{center}
    }
}

\setlength{\droptitle}{-2em}

  \title{Some Theory Underlying Simple Linear Regression}
    \pretitle{\vspace{\droptitle}\centering\huge}
  \posttitle{\par}
    \author{}
    \preauthor{}\postauthor{}
      \predate{\centering\large\emph}
  \postdate{\par}
    \date{2018-09-03}

\usepackage{booktabs}
\usepackage{longtable}
\usepackage{array}
\usepackage{multirow}
\usepackage[table]{xcolor}
\usepackage{wrapfig}
\usepackage{float}
\usepackage{colortbl}
\usepackage{pdflscape}
\usepackage{tabu}
\usepackage{threeparttable}
\usepackage{threeparttablex}
\usepackage[normalem]{ulem}
\usepackage{makecell}

\usepackage{amsthm}
\usepackage{xcolor}
\usepackage{xfrac}
\usepackage[framemethod=tikz]{mdframed}
\usepackage{graphicx}
\usepackage{rotating}
\usepackage{booktabs}
\usepackage{caption}
\definecolor{umn}{RGB}{153, 0, 85}
\definecolor{umn2}{rgb}{0.1843137, 0.4509804, 0.5372549}
\definecolor{myorange}{HTML}{EA6153}

\begin{document}
\maketitle

\captionsetup[figure]{labelfont=bf, textfont=it, width=.75\textwidth, justification=raggedright}
\captionsetup[table]{labelfont=bf, textfont=it, justification=raggedright, width=.75\textwidth}


% \mdfdefinestyle{proof}{
%    skipabove         = .5\baselineskip ,
%    skipbelow         = .5\baselineskip ,
%    leftmargin        = 20pt ,
%    rightmargin       = 10pt ,
%    innermargin       = 0pt ,
%    innerleftmargin   = 10pt ,
%    innerrightmargin  = 0pt ,
%    innerbottommargin = 0pt ,
%    linewidth         = 1pt,
%    linecolor         = gray,
%    topline           = true,
%    bottomline        = true,
%    rightline         = true
% }

% closed box instead of open box for the qed symbol:
% \newcommand*\closedbox{%
%    \leavevmode\hbox to.77778em{\hfil\rule{.675em}{.675em}\hfil}}
% \let\qedsymbol\closedbox

\let\qedsymbol\relax

\mdfdefinestyle{proof}{
   skipabove         = .5\baselineskip ,
   skipbelow         = .5\baselineskip ,
   leftmargin        = 0pt ,
   rightmargin       = 0pt ,
   skipabove         = 3em ,
   skipbelow         = 3em ,
   innermargin       = 0pt ,
   innertopmargin    = .5em ,
   innerleftmargin   = .5em ,
   innerrightmargin  = 0pt ,
   innerbottommargin = 0pt ,
   linecolor         = black,
   outerlinewidth    = 1pt,
%   hidealllines      = true ,
   % singleextra       = {
   %   \draw (O) -- ++(0,.675em) (O) -- ++(.675em,0) ;
   %   \draw (P-|O) -- ++(0,-.675em) (P-|O) -- ++(.675em,0) ;
   % },
   % firstextra        = {
   %   \draw (P-|O) -- ++(0,-.675em) (P-|O) -- ++(.675em,0) ;
   % },
   % secondextra       = {
   %   \draw (O) -- ++(0,.675em) (O) -- ++(.675em,0) ;
   % },
}

% put the new mdframed style around the proof environment:
\surroundwithmdframed[style=proof]{proof}



\theoremstyle{definition}
\newtheorem{definition}{Definition}[section]

\mdfdefinestyle{mystyle}{
  userdefinedwidth=5in, 
  align=center, 
  backgroundcolor=yellow, 
  roundcorner=10pt, 
  skipabove=2em
  }

\mdfdefinestyle{mystyle2}{
  userdefinedwidth=5.5in, 
  align=center, 
  skipabove=10pt, 
  topline=false, 
  bottomline=false, 
  linecolor=myorange, 
  linewidth=5pt
  }

\frenchspacing

To show that the slope estimator is unbiased, we need to show that
\(\mathbb{E}(B_1) = \beta\). We can express the slope estimator \(B_1\)
as a linear function of the observations, namely,

\[
B_1 = \sum m_iY_i \qquad \mathrm{where~} m_i = \frac{x_i-\bar{x}}{\sum(x_i-\bar{x})^2}
\]

We also make use of the fact that,

\[
\begin{split}
\sum(x_i-\bar{x})^2 &= \sum(x_i-\bar{x})(x_i-\bar{x})\\
&= \sum(x_i^2 -2x_i\bar{x} + \bar{x}^2) \\
&= \sum x_i^2 -2\bar{x} \sum x_i + \sum \bar{x}^2 \\
&= \sum x_i^2 -2\bar{x}(n\bar{x}) + n \bar{x}^2 \\
&= \sum x_i^2 -2n\bar{x}^2 + n \bar{x}^2 \\
&= \sum x_i^2 - n \bar{x}^2
\end{split}
\]

and the assumption of linearity in the population, namely that,

\[
\mathbb{E}(Y_i) = \beta_0 + \beta_1(x_i)
\]

Since \(B_1 = \sum \frac{(x_i-\bar{x})Y_i}{\sum x_i^2 - n \bar{x}^2}\),
then

\[
\begin{align*}
\mathbb{E}(B_1) &= \mathbb{E}\bigg(\sum \frac{(x_i-\bar{x})Y_i}{\sum x_i^2 - n \bar{x}^2}\bigg) \\
&= \mathbb{E}\bigg(\frac{1}{\sum x_i^2 - n \bar{x}^2} \times \sum \bigg[(x_i-\bar{x})Y_i\bigg] \bigg) \\
&= \frac{1}{\sum x_i^2 - n \bar{x}^2} \times \mathbb{E}\bigg(\sum \bigg[(x_i-\bar{x})Y_i\bigg] \bigg) \\
&= \frac{1}{\sum x_i^2 - n \bar{x}^2} \times \sum\bigg(\mathbb{E} \bigg[(x_i-\bar{x})Y_i\bigg] \bigg) \tag*{Expected value of a sum is the sum of an expected value.} \\
&= \frac{1}{\sum x_i^2 - n \bar{x}^2} \times \sum\bigg((x_i-\bar{x})\mathbb{E} \big[Y_i\big] \bigg) \\
&= \frac{1}{\sum x_i^2 - n \bar{x}^2} \times \sum\bigg((x_i-\bar{x})\big[\beta_0 + \beta_1(x_i)\big] \bigg) \tag*{Use definition of linearity assumption.} \\
&= \frac{1}{\sum x_i^2 - n \bar{x}^2} \times \sum\bigg(x_i\beta_0-\bar{x}\beta_0 + \beta_1x_i^2 - \beta_1x_i\bar{x} \bigg) \\
&= \frac{1}{\sum x_i^2 - n \bar{x}^2} \times \beta_0\sum x_i - n\bar{x}\beta_0 + \beta_1 \sum x_i^2 - \beta_1\bar{x}\sum x_i \bigg) \tag*{Distribute sum and pull out constants} \\
&= \frac{1}{\sum x_i^2 - n \bar{x}^2} \times \beta_0n\bar{x} - n\bar{x}\beta_0 + \beta_1 \bigg( \sum x_i^2 - \bar{x}n\bar{x} \bigg) \\
&= \frac{1}{\sum x_i^2 - n \bar{x}^2} \times \beta_1 \bigg( \sum x_i^2 - n\bar{x}^2 \bigg) \\
&= \beta_1
\end{align*}
\]


\end{document}
